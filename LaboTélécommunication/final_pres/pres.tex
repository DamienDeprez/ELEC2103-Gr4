\documentclass[11pt]{beamer}
\usetheme{Warsaw}
\usepackage[utf8]{inputenc}
\usepackage[french]{babel}
\usepackage[T1]{fontenc}
\usepackage{amsmath}
\usepackage{amsfonts}
\usepackage{amssymb}
\usepackage{graphicx}
%%\input{lib.tex}
\title{LELEC2103 - Projet 3 in Electricity}
\subtitle[\ldots]{Result of the lab 5 and lab 6}
\author[D. Deprez\and B. Ouachalih]{Damien Deprez\and Bilal Ouachalih}
\institute{EPL}
\date{16th december 2016}

\begin{document}

% TITLE PAGE
{
	\setbeamertemplate{headline}{}  
	\setbeamertemplate{footline}{}
	\setbeamertemplate{navigation symbols}{}
	\begin{frame}[noframenumbering]
		\titlepage
	\end{frame}
} 

% TABLE OF CONTENTS
{
	\setbeamertemplate{navigation symbols}{}
	\setbeamertemplate{headline}{}
	\begin{frame}[noframenumbering]{Plan de la présentation}
		\tableofcontents
	\end{frame}
}

\section{Frame detection and frequency offset correction}

\subsection{Moose algorithm}

\begin{frame}
\frametitle{Base}
\begin{itemize}
\item The received signal is
\begin{equation}
y[n] = hs[n-d] + v[n]
\end{equation}
with $h$ the channel coefficient and $d$ the frame offset.
\item The correlation with the training sequence is given by
\begin{equation}
R[n] = \left| \sum_{k=0}^{N_t-1} t^*[k]y[n+k] \right|^2
\end{equation}
with $\{t[n]\}_0^{N_t-1}$ is the known training sequence. In the lab, it's 4 length 11 Barker Sequence
\end{itemize}

\end{frame}


\subsubsection{Frequency estimation offset}
\begin{frame}
\frametitle{Sliding correlator}


\end{frame}

\subsection{Limit of the method}

\begin{frame}
\frametitle{Maximum frequency offset}


\end{frame}

\subsection{Result}
\subsubsection{}
\begin{frame}
\frametitle{Timing Static error VS RX Oversampling factor}

\end{frame}

\subsubsection{Constellation}

\begin{frame}
\frametitle{Impact of the oversampling factor at the receiver}


\end{frame}

% TABLE OF CONTENTS
{
	\setbeamertemplate{navigation symbols}{}
	\setbeamertemplate{headline}{}
	\begin{frame}[noframenumbering]{Plan de la présentation}
		\tableofcontents
	\end{frame}
}

\section{OFDM modulation}
\begin{frame}
\frametitle{Modulation}

\begin{figure}[!ht]
    \begin{minipage}[b]{0.48\linewidth}
        \centering \includegraphics[scale=0.6]{img/OFDDM_modulator.png}
     \caption{Block diagram of the OFDM modulator}
     \label{fig2}
    \end{minipage}\hfill
    \begin{minipage}[b]{0.48\linewidth}  
    \centering  
    \begin{itemize}
    \item[$\bullet$] N : number of subcarriers
    \item[$\bullet$] K : number of null subcarriers
    \item[$\bullet$] $L_c$ : cyclic prefix (against ISI)
    \item[$\bullet$] Block-based modulation technique
    \end{itemize}
        
    \end{minipage}
\end{figure}
\begin{equation}
w[n]=\frac{1}{N} \sum_{m=0}^{N-1} s[m]\exp^{j2\pi\frac{m(n-L_c)}{N}}
\end{equation}
\end{frame}

\begin{frame}
\frametitle{Demodulation}
\begin{figure}[!ht]
         \centering \includegraphics[scale=0.75]{img/OFDDM_demodulator.png}
 \caption{Block diagram of the OFDM demodulator}\label{fig3}  
\end{figure}
\begin{equation}  
 y[n]=\sum_{l=0}^{L} h[l]w[n-l]+v[n]
 \end{equation}
\end{frame}

\subsection{Experiment using OFDM}
\subsubsection{Results}

\begin{frame}
\frametitle{Symbol rate}
\begin{center}
	\begin{tabular}{c|c|c}
		  & Narrowband & Wideband\\
		  \hline
	Tx Sample rate & 4 $MSample/s$ & 20 $MSample/s$ \\	  
	Tx Oversample factor & 20 & 4\\
	\textbf{ODFM symbol rate} &  $\textbf{2777.77 symbol/s}$ & $\textbf{69444.44 symbol/s}$ \\ 
	\hline
     & \color{red} \textbf{Flat}  & \color{red} \textbf{Frequency-selective}\\                               
	\end{tabular}
	\label{tab1}
\end{center}
\begin{itemize}
\item[$\bullet$] Larger symbol rate in wideband than in narrrowband.
\item[$\bullet$] Flat-fading : Always the same attenuation on the signal
\item[$\bullet$] Frequency-selective : Different attenuation on the signal
\end{itemize}

\end{frame}

\begin{frame}
\frametitle{Power delay of the channels}

\begin{figure}[!ht]
    \begin{minipage}[b]{0.48\linewidth}
        \centering \includegraphics[scale=0.35]{img/power_delay_narrow.png}
     \label{fig4}
    \end{minipage}\hfill
    \begin{minipage}[b]{0.48\linewidth}
         \centering \includegraphics[scale=0.35]{img/power_delay_wideband.png}
    \end{minipage}
\end{figure}
\begin{itemize}
\item[$\bullet$] Narrowband power delay < Wideband power delay  
\end{itemize}
\end{frame}
\subsubsection{Narrowband VS Wideband Channel}
\begin{frame}
\frametitle{Frequency response of each channels}

\begin{figure}[!ht]
    \begin{minipage}[b]{0.48\linewidth}
        \centering \includegraphics[scale=0.33]{img/channel_response_narrow.png}
     \label{fig2}
    \end{minipage}\hfill
    \begin{minipage}[b]{0.48\linewidth}
         \centering \includegraphics[scale=0.33]{img/channel_response_wideband.png}
    \end{minipage}
\end{figure}
\begin{itemize}
\item[$\bullet$] Flat response in narrowband
\item[$\bullet$] Frequency-selective response in wideband 
\item[$\bullet$] No DC component due to RF distorsion
\end{itemize}

\end{frame}
\subsection{Frequency selectivity of wireless channel}
\begin{frame}
\frametitle{Frequency-selectivity of the channel}

\begin{figure}[!ht]
    \begin{minipage}[b]{0.48\linewidth}
        \centering \includegraphics[scale=0.41]{img/multicarrier_200hz.png}
     \label{fig6}
    \end{minipage}\hfill
    \begin{minipage}[b]{0.48\linewidth}
         \centering \includegraphics[scale=0.35]{img/SingleCarrier_Offset_200}
\label{fig7}
    \end{minipage}
\end{figure}

\begin{itemize}
\item[$\bullet$] In OFDM, differents attenuations on each symbols (Frequency-selective)
\end{itemize}
\end{frame}

\begin{frame}
\frametitle{N=64 VS N=1024}
The subcarrier spacing is defined like
\begin{equation}
\Delta_c=\frac{1}{NT} 
\end{equation}
with T the sample period and N the FFT size

\begin{center}
	\begin{tabular}{c|c|c}
		  & $N=64$ & $N=1024$\\
		  \hline
	$\Delta_c$ & $\textbf{15625~kHz}$ & $\textbf{976.56 kHz}$ \\
	\end{tabular}
	\label{tab3}
\end{center}

\begin{itemize}

\item[$\bullet$] Less subcarrier spacing for larger N
\item[$\bullet$] Effect : ICI (inter-carrier interference)
\end{itemize}


\end{frame}



\section{Multi-carrier VS Single-carrier modulation}

\begin{frame}
\frametitle{Multi VS Single carrier}
\begin{minipage}[t]{0.48\linewidth}
OFDM
\begin{itemize}
\item[$\bullet$] {\color{red} ICI}
\item[$\bullet$] {\color{red} Frequency offset}
\item[$\bullet$] For frequency-selective channel
\item[$\bullet$] Need more power
%%\item[$\bullet$] Less complexity

\end{itemize}
\end{minipage}\hfill
\begin{minipage}[t]{0.48\linewidth}
Single-carrier
\begin{itemize}
\item[$\bullet$] {\color{red} ISI}
\item[$\bullet$] {\color{red} Delay} 
\item[$\bullet$] For flat channel
\item[$\bullet$] Need less power 

\end{itemize}

\end{minipage}





\end{frame}





\end{document}
