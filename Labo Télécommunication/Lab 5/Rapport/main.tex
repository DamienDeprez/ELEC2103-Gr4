\documentclass[frenchb, oneside, headings=normal]{scrartcl}

\input{lib.tex}

\usepackage{epstopdf}
\usepackage{wrapfig}
\begin{document}

\title{Projet ELEC Master 1 - Labo 5}
\subtitle{Groupe 4}
\author{Deprez Damien \and Bilal Ouachalih }
\date{18 november 2016}
\maketitle

The purpose of this lab is to correct the frequency offset and the delay due to the channel.

\section{Questions of the Pre-Lab}

\subsection{Describe what happens to the error rate of your system when the estimate for the delay in the channel is off by more than one symbol time}

First, we modified a little bit the schematic of the SlidingCorrelation.vi. In fact, we have added a controller to make an error on the delay to let us observe what happened. Here is the result\\

\begin{figure}[!ht]
  \begin{minipage}[b]{0.48\linewidth}
        \centering \includegraphics[scale=0.45]{img/Sliding_correletaion_OFF_AWGN_5dB_shift_bit_0.png}
    \caption{Error rate for a delay+0 in AWGN}
    \label{fig1}
    \end{minipage}\hfill
    \begin{minipage}[b]{0.48\linewidth}
         \centering \includegraphics[scale=0.45]{img/Sliding_correletaion_OFF_AWGN_5dB_shift_bit_1.png}
          \caption{Error rate for a delay+1 in AWGN}
          \label{fig2}
    \end{minipage}
\end{figure}

\begin{figure}[!ht]
    \begin{minipage}[b]{0.48\linewidth}
        \centering \includegraphics[scale=0.45]{img/Sliding_correletaion_OFF_AWGN_5dB_shift_bit_2.png}
     \caption{Error rate for a delay+2 in AWGN}
     \label{fig3}
    \end{minipage}\hfill
    \begin{minipage}[b]{0.48\linewidth}
         \centering \includegraphics[scale=0.45]{img/Sliding_correletaion_OFF_AWGN_5dB_shift_bit_3.PNG}
          \caption{Error rate for a delay+3 in AWGN}
          \label{fig4}
    \end{minipage}  

\end{figure}
\begin{figure}[!ht]
    \begin{minipage}[b]{0.48\linewidth}
        \centering \includegraphics[scale=0.45]{img/Sliding_correletaion_OFF_AWGN_5dB_shift_bit_4.PNG}
     \caption{Error rate for a delay+4 in AWGN}
     \label{fig5}
    \end{minipage}\hfill
    \begin{minipage}[b]{0.48\linewidth}
         \centering \includegraphics[scale=0.45]{img/Sliding_correletaion_OFF_AWGN_5dB_shift_bit_5.PNG}
 \caption{Error rate for a delay+5 in AWGN}\label{fig6}
    \end{minipage}
\end{figure}

As we can seen on figure \ref{fig1},\ref{fig2},\ref{fig3} and \ref{fig4}, we have the error rate which is yet acceptable for a added delay of 0 to 3. But on figure \ref{fig5} and \ref{fig6}, when we add to the estimated delay, 4, we have a bit error rate of $\simeq 0.5$, which is very bad. Which means that one bit on two is not welle placed.

\subsection{Describe what happens to the received constellation if you do not correct for the 201 Hz frequency offset}

When we set the controller of the frequency offset on FALSE, we don't correct the error of frequency anymore, and we obtain the following result

\begin{figure}[!ht]
\centering
\includegraphics[scale=0.7]{img/test_offset_201hz_OFF.PNG}
\caption{Constellation for a frequency offset of $201$ Hz without correction}
\label{freq_correct_off}
\end{figure}

On the figure \ref{freq_correct_off}, we can see  that we don't have a nice constellation with 4 clear points in the 4 corner, but multiple point distributed on a portion of circle. 


\newpage

\subsection{what are the range of frequency offsets that you can estimate/correct using the Moose algorithm?}

If we based our observation on waht is said in the book, we can determine the range of frequency to have a good work of the Moose.vi. In fact, we have this condition

\begin{equation}
|\epsilon| \leq \frac{1}{2*N_t} = \frac{1}{2*44} = 0.01136
\label{cdt1}
\end{equation}

Or in an other form 

\begin{equation}
|f_e| \leq \frac{1}{2*T*N_t} = \frac{1}{2*1e-6*44} \simeq 11 kHz
\label{cdt2}
\end{equation}\\

So, if we have a frequency offset greater then $11 kHz$, the Moose method will not work anymore. In order the check if we are write, we wade two simulation. The first one with a frequency offset of $10 kHz$ and an other with $12 kHz$, and the result is interesting.\\

\begin{figure}[!ht]
    \begin{minipage}[b]{0.48\linewidth}
        \centering \includegraphics[scale=0.45]{img/test_Offset_10k_OK_limitMooseCheck.png}
     \caption{Simulation with a frequency offset of $10 kHz$}
     \label{MooseLimit1}
    \end{minipage}\hfill
    \begin{minipage}[b]{0.48\linewidth}
         \centering \includegraphics[scale=0.45]{img/test_Offset_12k_OK_limitMooseCheck.png}
\caption{Simulation with a frequency offset of $12 kHz$}
 \label{MooseLimit2}
    \end{minipage}
\end{figure}

On the figure \ref{MooseLimit1} and \ref{MooseLimit2}, the result speaks of himself. Moose method doesn't work anymore above $\simeq 11 kHz$.

\section{Question of the Lab}

\subsection{Before changing any of the settings, calculate an average value (using five runs or so) of the inherent offset between the transmitter and receiver.}

%\textbf{Si tu sais le faire Damien avec les valeurs que tu as prises au labo}
We run five times the reciever programm and we obtains the following result 
\begin{center}
	\begin{tabular}{c|c}
		run & frequency offset [\si{\hertz}]\\
		\hline
		1 & 7.75\\
		2 & 12.74\\
		3 & 5.73\\
		4 & -5.26\\
		5 & -1.158\\
	\end{tabular}
\end{center}
The average offset between the transmitter and receiver is $3.96$ \si{\hertz}. 


\subsection{Based on the system parameters, what is the range of frequency offsets that can be estimated by your frequency offset estimation algorithm?}

Like in the previous section, we use the equation \ref{cdt1} and \ref{cdt2}, to determine the range of frequency. In the case of USRP transmission, the range of admissible frequency offset is between $0$ and $2272 Hz$.

\newpage

\subsection{Let fM be the maximum correctable frequency offset of your system (i.e., as calculated in the previous question). Modify the carrier frequency of your transmitter so that you will cause an effective offset of 0.80fM at the receiver. What is the new value of the carrier frequency at your transmitter?}

If we make a test with the USRP, and we add to the carrier frequency $0.8*f_m=0.8*2272=1818 Hz$, we obtained the following result

\begin{figure}[!ht]
  \begin{minipage}[b]{0.48\linewidth}
        \centering \includegraphics[scale=0.7]{img/USRP_carrieroffset_227.PNG}
    \caption{Result for a frequency offset of $227 Hz$}
    \label{fig8}
    \end{minipage}\hfill
    \begin{minipage}[b]{0.48\linewidth}
         \centering \includegraphics[scale=0.7]{img/USRP_carrieroffset_1818.PNG}
          \caption{Result for a frequency offset of $1818 Hz$}
          \label{fig9}
    \end{minipage}
\end{figure}

On figures \ref{fig8} and \ref{fig9}, We can observe that, more we increase the frequency offset near the maximum one, more the point on the constellation are not perfectly centered in the 4 corner of the figure.

\begin{figure}[!ht]
    \begin{minipage}[b]{0.48\linewidth}
        \centering \includegraphics[scale=0.7]{img/USRP_value_227.PNG}
     \caption{Result data for a frequency offset of $227 Hz$}
     \label{fig10}
    \end{minipage}\hfill
    \begin{minipage}[b]{0.48\linewidth}
         \centering \includegraphics[scale=0.7]{img/USRP_value_1818.PNG}
          \caption{Result data for a frequency offset of $1818 Hz$}
          \label{fig11}
    \end{minipage}  
\end{figure}

The figures \ref{fig10} and \ref{fig11} show the result given by the topRX.vi, and show that, the frequency offset inserted at the transmitter is very similar to the one he measure at the receiver. Which means, that the channel itself doesn't introduce a significative error in frequency, but more in the delay.






\end{document}
