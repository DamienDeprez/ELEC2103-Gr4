\documentclass[frenchb, oneside, headings=normal]{scrartcl}

\input{lib.tex}

\usepackage{epstopdf}
\usepackage{wrapfig}
\begin{document}

\title{Projet ELEC Master 1 - Labo 4}
\subtitle{Groupe 4}
\author{Deprez Damien \and Bilal Ouachalih }
\date{21 octobre 2016}
\maketitle


\section{Questions of the Pre-Lab}

\subsection{In your implementation of toeplitz.vi you were required to build a Toeplitz matrix given the initial row and column of the matrix. Notice the first element of row and column should be equal. What will
your VI do if the initial element of each array is different?} 


For instance, if I enter the following value in the toeplitz.vi, here is what we have\\
\begin{itemize}
\item If the row is $\begin{pmatrix}1 & 2 & 3\\ \end{pmatrix}$
	  and the column is $\begin{pmatrix}1\\4\\6 \end{pmatrix}$
 	  the Toeplitz matrix is $\begin{pmatrix}1 & 2 & 3\\4 & 1 & 2\\6 & 4 & 1
      \end{pmatrix}$\\

\item Now, If the row is $\begin{pmatrix}2 & 3 & 5\\ \end{pmatrix}$
	  and the column is $\begin{pmatrix}1\\4\\6 \end{pmatrix}$
 	  the Toeplitz matrix is $\begin{pmatrix}1 & 3 & 5\\4 & 1 & 3\\6 & 4 & 1
      \end{pmatrix}$.\newline \\ As we can see, the first value of the column     	  has the priority in the Toeplitz matrix.

\end{itemize}
\subsection{Observe how the bit-error rate performance
of your equalizer varies with SNR for various equalizer lengths. Plot average BER as a function of SNR for Lf + 1 = 1 and Lf + 1 = 6.
Vary SNR from 0 dB to 14 dB in increments of 2 dB}

\begin{figure}[ht!]
\centering
\includegraphics[scale=0.25]{img/SNR.png}
\caption{Bit-error rate VS SNR}
\label{BER}
\end{figure}


As we can see on the figure \ref{BER}, the bit-error rate is decreasing when the SNR increase(in absolute value).
Moreover, the erro is much less with an equalizer length of 6 then  of 1


\subsection{What happens to the received signal constellation when you set the equalizer length to one?}

After a little bit of thinking, we can make the assumption that if we increase the length of the equalizer, the result will become better and better. In fact, we have the following theorical formula. 

\begin{equation}
\sum_{l=0}^{L_f}f[l]\hat{h}[n-l]\approx \delta [n-n_d]
\label{equ1}
\end{equation}

As we can see in the equation \ref{equ1}, we are estimating the parameter $f$, the filter which is being used to remove the interference symbol of the received signal coming from the channel canal. And if we increase the size of terms for the filter, we will have a better result. Now, we can validate that assumption thanks to the simulation. Here is the result.

\begin{figure}[ht]
    \begin{minipage}[b]{0.48\linewidth}
        \centering \includegraphics[scale=0.6]{img/ISI-1.png}
    \caption{Constellation for an equalizer length of 1}
    \label{fig1}
    
    \end{minipage}\hfill
    \begin{minipage}[b]{0.48\linewidth}
         \centering \includegraphics[scale=0.6]{img/ISI-2.png}
        \caption{Constellation for an equalizer length of 2}
        \label{fig2}
    \end{minipage}
    \begin{minipage}[b]{0.48\linewidth}
        \centering \includegraphics[scale=0.6]{img/ISI-3.png}
    \caption{Constellation for an equalizer length of 3}
    \label{fig3}
    
    \end{minipage}\hfill
    \begin{minipage}[b]{0.48\linewidth}
         \centering \includegraphics[scale=0.6]{img/ISI-4.png}
        \caption{Constellation for an equalizer length of 4}
        \label{fig4}
    \end{minipage}
    
    \begin{minipage}[b]{0.48\linewidth}
        \centering \includegraphics[scale=0.6]{img/ISI-5.png}
    \caption{Constellation for an equalizer length of 5}
    \label{fig5}
    
    \end{minipage}\hfill
    \begin{minipage}[b]{0.48\linewidth}
         \centering \includegraphics[scale=0.6]{img/ISI-6.png}
        \caption{Constellation for an equalizer length of 6}
        \label{fig6}
    \end{minipage}
\end{figure}

After a short observation of the figure \ref{fig1} to \ref{fig6}, we can check the assumption that has been done before is correct. 


\newpage
\section{Question for the lab}

\subsection{What is the symbol rate of your system?}

First we use the following formula

\begin{equation}
T_s=N*T_N in [Symbol/s]
\end{equation}

with N which is the oversampling factor and $T_n$, the sample rate. At the end, the symbol rate must remain constant. So, for $T_N=2M Sample/s$, and $N=20$, \textbf{$T_s=40 MSymbol/s$}

\subsection{What is the passband bandwidth of your system?}

The bandwidth is 40 MHz.

\subsection{Based on your observations, describe the impairments imparted to the received constellation}

\begin{figure}[ht!]
\centering
\includegraphics[scale=0.9]{img/USRP-Rotate.png}
\caption{Constellation of the received signal when the channel of the USRP is being used}
\label{usrp}
\end{figure}

On the figure \ref{usrp}, we can observe 2 things:

\begin{itemize}

\item First, the constellation has rotated. this phenoma could maybe come from a phase shift introduced in the received signal through the channel canal. We could make a link with the previous lab, which consisted of estimate the phase shift.

\item Second, the amplitude has decreased. In fact, in theory, for a QAM constellation, the magnitude is $\approx 0.7 [dB]$. On the figure \ref{usrp}, we have a magnitude of $\approx 0.4 [dB]$. We can deduce that we have an attenuation due to the canal, in other word a loss of power of the signal.



\end{itemize}


   






\end{document}
