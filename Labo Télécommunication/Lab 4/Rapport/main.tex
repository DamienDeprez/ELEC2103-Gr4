\documentclass[frenchb, oneside, headings=normal]{scrartcl}

\input{lib.tex}

\usepackage{epstopdf}
\begin{document}

\title{Projet ELEC Master 1 - Labo 4}
\subtitle{Groupe 4}
\author{Deprez Damien \and Bilal Ouachalih }
\date{21 octobre 2016}
\maketitle


\section{Questions of the Pre-Lab}

\subsection{In your implementation of toeplitz.vi you were required to build a Toeplitz matrix given the initial row and column of the matrix. Notice the first element of row and column should be equal. What will
your VI do if the initial element of each array is different?} 


For instance, if I enter the following value in the toeplitz.vi, here is what we have\\
\begin{itemize}
\item If the row is $\begin{pmatrix}1 & 2 & 3\\ \end{pmatrix}$
	  and the column is $\begin{pmatrix}1\\4\\6 \end{pmatrix}$
 	  the Toeplitz matrix is $\begin{pmatrix}1 & 2 & 3\\4 & 1 & 2\\6 & 4 & 1
      \end{pmatrix}$\\

\item Now, If the row is $\begin{pmatrix}2 & 3 & 5\\ \end{pmatrix}$
	  and the column is $\begin{pmatrix}1\\4\\6 \end{pmatrix}$
 	  the Toeplitz matrix is $\begin{pmatrix}1 & 3 & 5\\4 & 1 & 3\\6 & 4 & 1
      \end{pmatrix}$.\newline \\ As we can see, the first value of the column     	  has the priority in the Toeplitz matrix.

\end{itemize}

\subsection{What happens to the received signal constellation when you set the equalizer length to one? Describe what happens to the constellation as you vary the equalizer length from one to six}

After a little bit of thinking, we can make the assumption that if we increase the length of the equalizer, the result will become better and better. In fact, we have the following theorical formula. 

\begin{equation}
\sum_{l=0}^{L_f}f[l]\hat{h}[n-l]\approx \delta [n-n_d]
\label{equ1}
\end{equation}

As we can see in the equation \ref{equ1}, we are estimating the parameter $f$, which is the filter which is being used to remove the interference symbol of the received signal coming from the channel canal.
Now, we can validate that assumption thanks to the simulation. Here is the result.


%ajouter les 6 constellation.(ISS-1->ISS-6)

\subsection{Observe how the bit-error rate performance
of your equalizer varies with SNR for various equalizer lengths. Plot
average BER as a function of SNR for Lf + 1 = 1 and Lf + 1 = 6.
Vary SNR from 0 dB to 14 dB in increments of 2 dB}

%add graph of the BER VS SNR for Lf=1 et 6


\section{Question for the lab}

\subsection{What is the symbol rate of your system?}

First we use the following formula

\begin{equation}
T_s=N*T_N in [Symbol/s]
\end{equation}

with N which is the oversampling factor and $T_n$, the sampling rate. In fact, the symbol rate must remain constant.

So, for $T_N$=5M symbol/s.

\subsection{What is the passband bandwidth of your system?}

The bandwidth is 40 MHz.

\subsection{Based on your observations, describe the impairments imparted to the received constellation}





   






\end{document}
