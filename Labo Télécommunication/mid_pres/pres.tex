\documentclass[11pt]{beamer}
\usetheme{Warsaw}
\usepackage[utf8]{inputenc}
\usepackage[french]{babel}
\usepackage[T1]{fontenc}
\usepackage{amsmath}
\usepackage{amsfonts}
\usepackage{amssymb}
\usepackage{graphicx}
%%\input{lib.tex}
\title{LELEC2103 - Projet 3 in Electricity}
\subtitle[\ldots]{Result of the lab 3 and lab 4}
\author[D. Deprez\and B. Ouachalih]{Damien Deprez\and Bilal Ouachalih}
\institute{EPL}
\date{28th october 2016}

\begin{document}

% TITLE PAGE
{
	\setbeamertemplate{headline}{}  
	\setbeamertemplate{footline}{}
	\setbeamertemplate{navigation symbols}{}
	\begin{frame}[noframenumbering]
		\titlepage
	\end{frame}
} 

% TABLE OF CONTENTS
{
	\setbeamertemplate{navigation symbols}{}
	\setbeamertemplate{headline}{}
	\begin{frame}[noframenumbering]{Plan de la présentation}
		\tableofcontents
	\end{frame}
}

\section{Symbol timing recovery}

\subsection{Maximum Energy}

\begin{frame}
\frametitle{Maximum Energy}

\begin{itemize}
\item The received signal after filtering is 
\begin{equation}
y[n] = \sqrt{E_x}\alpha e^{j\phi} [s[n]g(\tau_d)+\sum_{m \neq n} s[m] g((m-n)T-\tau_d)]+ v[n]
\end{equation}
with $\tau_d$ is the symbol interference

\item The output energy is given by
\begin{equation}
J(\tau) = E|(y(nT+\tau)|^2= \alpha^2 E_x \sum_m |g(mT+\tau-\tau_d)|^2 + \sigma_v ^2
\end{equation}

We define $\hat{\tau}=argmax_{\tau} J(\tau)$

\end{itemize}   

\end{frame}


\subsubsection{Direct Maximum Energy}
\begin{frame}
\frametitle{Direct Maximum Energy}

\begin{itemize}

\item In descrete time, we have $\hat{\tau} = \frac{kT}{M} \ \ \ k \in  [0..M-1]$ and so
		
\begin{equation} 
J[k] = E|r(nT+\frac{kT}{M})|^2
\end{equation}

\item For P symbols
\begin{equation}
J_{approx}[k] = \frac{1}{P} \sum \limits_{p=0}^{P-1} |r(pT+\frac{kT}{M})|^2
\label{equ1}
\end{equation}

We have the maximum energy for $\hat{k}=argmax\ J_{approx}[k]$


\end{itemize}

\end{frame}

\subsubsection{Earlygate}

\begin{frame}
\frametitle{Earlygate}

\begin{itemize}

\item We will try to find the maximum by canceling the derivative

\begin{equation}
\frac{d}{d\tau} J(\tau)  \simeq \frac{1}{P} \sum_{p=0}^{P-1} 2Re(r(pT+\tau)(r^*(pT+\tau+\delta)-r^*(pT+\tau-\delta)))
\end{equation}

We take $\delta$ multiple of $T/M$, here equal to one

\item At the end, we obtain  

\begin{equation}
\hat{k}= {arg\min}_{k=0,..,M-1} J_{\delta}[k]
\end{equation}

\end{itemize}

\end{frame}

\subsection{Result}
\subsubsection{Timing Static Error}
\begin{frame}
\frametitle{Timing Static error VS RX Oversampling factor}
\begin{figure}
\includegraphics[width=0.9\textwidth]{img/Timing-Static-Error.png}
\end{figure}

\end{frame}

\subsubsection{Constellation}

\begin{frame}
\frametitle{Impact of the oversampling factor at the receiver}

\begin{figure}
   \begin{minipage}[c]{.46\linewidth}
   \centering
   {\large Constellation for N=2}
            \tiny
      \includegraphics[width=1\textwidth]{img/ConstellationN2-MaxEnergy.png}
      
   \end{minipage} \hfill
   \begin{minipage}[c]{.46\linewidth}
   \centering
   {\large Constellation for N=20}
            \tiny
      \includegraphics[width=1\textwidth]{img/ConstellationN20-MaxEnergy.png}
   \end{minipage}
\end{figure}

\end{frame}

\begin{frame}
\frametitle{Effect of M on the Max Energy method}

\begin{itemize}

\item If M is increasing => Result at the receiver  is better. Why ?

\item In equation \ref{equ1}, if M is incresing, 

\end{itemize}

\end{frame}


\section{Channel Estimation and Equalization}

\subsection{Purpose of the lab}
\begin{frame}
Channel canal
\begin{equation}
z(t) = \alpha_0 e^{j\phi_0}x(t-\tau_0) + \alpha_1 e^{j\phi_1}x(t-\tau_1) + v(t)
\end{equation}
Direct Least-Squares Equalizer
\begin{equation}
t[n] = \sum_{l=0}^{L_f} {f_{n_d}[l]y[n + n_d -l ]}
\end{equation}
\end{frame}
\subsection{Result}
\subsubsection{Effect of the equalizer length}
\begin{frame}
\frametitle{Impact of the equlizer length at the reciever}
\begin{minipage}[c]{0.46\linewidth}
\centering 
{Constellation for $L_f + 1 = 2$}
\includegraphics[width=\textwidth]{img/ISI-2}
\end{minipage}
\begin{minipage}[c]{0.46\linewidth}
\centering 
{Constellation for $L_f + 1 = 6$}
\includegraphics[width=\textwidth]{img/ISI-6}
\end{minipage}
\end{frame}
\subsubsection{Effect of the noise}
\begin{frame}
\frametitle{Impact of the noise at the reciever}
\includegraphics[width=.9\linewidth]{img/SNR}
\end{frame}
\subsubsection{Experiment on the USRP}
\begin{frame}
\frametitle{Experiment on the USRP}
\includegraphics[width=.9\linewidth]{img/USRP-Rotate}
\end{frame}
\end{document}
