\documentclass[frenchb, oneside, headings=normal]{scrartcl}

\input{lib.tex}

\usepackage{epstopdf}
\usepackage{wrapfig}
\usepackage{verbatim}
\begin{document}

\title{Projet ELEC Master 1 - Labo 6}
\subtitle{Groupe 4}
\author{Deprez Damien \and Bilal Ouachalih }
\date{2 december 2016}
\maketitle

The purpose of this lab is to study the OFDM (\textit{\textbf{orthogonal frequency division multiplexing}}) modulation. An interpretation is that OFDM devides a frequency selective channel into N flat fadding subchannels. We can see that as N parallel flat fadding channels multiplexed in the frequency domain. The cyclic prefix is used to avoid intersymbol interference (\textbf{ISI}).

\begin{figure}[!ht]
\centering
\includegraphics[scale=0.8]{img/operation_ofdm.png}
\caption{Representation of what's being done by the OFDM modulation}
\label{fig1}
\end{figure}

\section{Pre-lab}

In the first part of the lab, we have implemented an OFDM modulator  and the corresponding demodulator. First, at the transmitter side

\begin{itemize}

\item We have a sequence of $N-K$ symbols where N is the number of subcarriers and K is the number of null subcarrier.

\item \textbf{Null symbols} are inserted into groups of $N-K$ transmit suymbols.

\item We have an \textbf{IFFT} of the previous symbols.

\item Then, a \textbf{cyclic prefix} of length $L_c$ is added at the transmitter.

\end{itemize}

Second, at the receiver side

\begin{itemize}

\item The receiver consider block of $N+L_c$ without regarding the first $L_c$ samples of each block.

\item Then, we take back the \textbf{FFT} of the N symbols.

\end{itemize}

This previous 'operations' are represented on the block diagam on the figure \ref{fig2} and \ref{fig3}.

\begin{figure}[!ht]
    \begin{minipage}[b]{0.48\linewidth}
        \centering \includegraphics[scale=0.9]{img/OFDDM_modulator.png}
     \caption{Block diagram of the OFDM modulator}
     \label{fig2}
    \end{minipage}\hfill
    \begin{minipage}[b]{0.48\linewidth}
         \centering \includegraphics[scale=0.9]{img/OFDDM_demodulator.png}
 \caption{Block diagram of the OFDM demodulator}\label{fig3}
    \end{minipage}
\end{figure}

\section{Lab experiment}

\subsection{what is the respective OFDM symbol rate in the narrowband and wideband systems you have set up ?}

For $N=64$, and $L_c=8$

\begin{center}
	\begin{tabular}{c|c|c}
		  & Narrowband & Wideband\\
		  \hline
	Tx Sample rate & 4 $MSample/s$ & 20 $MSample/s$ \\	  
	Tx Oversample factor & 20 & 4\\
	\textbf{ODFM symbol rate} &  $\textbf{2777.77 symbol/s}$ & $\textbf{69444.44 symbol/s}$ \\ 
	\end{tabular}
	\label{tab1}
\end{center}

We simply have calculated the OFDM symbol rate by using the following formula

\begin{equation}
OFDM~symbol~rate = \frac{Tx~sample~rate}{(N+L_c)*Tx~oversample~factor}
\end{equation}

We can directly see that the symbol period is greater in a narrowband channel the in an wideband channel. (important for the future question). 

\subsection{What are the effective lengths of the narrowband and wideband channels respectively?}

If we take a look on the figure \ref{fig4} and \ref{fig5} representing the power delay in both case (narrow and wideband channel), we can deduce the effective length of the channel. The effective length are of $\textbf{2.5e-7~s}$ and $\textbf{1e-7~s}$ respectively for the narrowband and wideband channel.

\begin{figure}[!ht]
    \begin{minipage}[b]{0.48\linewidth}
        \centering \includegraphics[scale=0.45]{img/power_delay_narrow.png}
     \caption{Power delay of the narrowband channel}
     \label{fig4}
    \end{minipage}\hfill
    \begin{minipage}[b]{0.48\linewidth}
         \centering \includegraphics[scale=0.45]{img/power_delay_wideband.png}
 \caption{Power delay of the wideband channel}\label{fig5}
    \end{minipage}
\end{figure}
\subsection{Describe the frequency responses of each channel.}

\begin{figure}[!ht]
    \begin{minipage}[b]{0.48\linewidth}
        \centering \includegraphics[scale=0.45]{img/channel_response_narrow.png}
     \caption{Response of a narrowband channel}
     \label{fig2}
    \end{minipage}\hfill
    \begin{minipage}[b]{0.48\linewidth}
         \centering \includegraphics[scale=0.45]{img/channel_response_wideband.png}
 \caption{Response of a wideband channel}\label{fig3}
    \end{minipage}
\end{figure}

We can characterize a flat or a frequency-selective channel by comparing the bandwidth of the signal with the bandwidth of the channel (or the symbol period and the effective length).\\

\begin{itemize}
\item For a flat channel, we have that \textbf{Bandwidth of channel > Bandwidth of the symbol}.
\item For a frequency-selective channel, we have just the inverse, \textbf{Bandwidth of channel < Bandwidth of the symbol}. 

\end{itemize}

By using the value computed in the previous question, we can deduce that

\begin{center}
	\begin{tabular}{c|c|c}
		  & Narrowband & Wideband\\
		  \hline
	Tx Sample rate & 4 $MSample/s$ & 20 $MSample/s$ \\	  
	Tx Oversample factor & 20 & 4\\
	\hline
	\textbf{Bandwidth of the channel} &   & \\
    \textbf{Bandwidth of the symbol} & $\textbf{0.2~MHz}$ & $\textbf{5~Mhz}$\\
        \hline
                                     & \color{red} \textbf{Flat}  & \color{red} \textbf{Frequency-selective}\\                               
	\end{tabular}
	\label{tab1}
\end{center}

Another thing we can see from the channel response, is that we have a big attenuation at $0 Hz$. Which means that the zero frequency or DC is commonly nulled due to RF distortion at DC. 

\subsection{Show that in an OFDM system, when $L_h$ = 0 the frequency response of the channel is necessarily flat fading, and when $L_h > 0$, the channel is frequency-selective}

We consider the multipath channel model in absence of noise 
\begin{equation}
y[n]=\sum_{l=0}^{L_h} h[l]x[n-l] 
\end{equation}

\begin{enumerate}

\item If $L_h=0$, we can rewrite the expression above as
\begin{equation}
y[n]=h[0]x[n]
\end{equation}
And we want to show that the frequency response is the same for all subchannels. ($H[n]=H[m]~\forall n,m$)
\begin{equation}
H[k]=h[0]\exp^{\frac{-2\pi jk*\color{red} 0}{N}}=h[0]~\forall k
\end{equation} 

\item if $L_h>0$, we will want to show that $H[n] \neq H[m]$ for at least one different n, m.

First, we take the Fourrier transform of h[l].
\begin{equation}
H[k]=\sum_{l=0}^{L_h} h[l]\exp^{\frac{-2\pi jkl}{N}}~\forall k
\end{equation}

Now, we are going to consider 2 values of k. (0 and N/2)
\begin{equation}
H[0]=\sum_{l_1=0}^{L_h} h[l_1]*1=\sum_{l_1=0}^{L_h} h[l_1]\\
\end{equation}
\begin{equation}
H[N/2]=\sum_{l_2=0}^{L_h} h[l_2]\exp^{\frac{-2\pi jNl_2}{2N}}=\sum_{l_2=0}^{L_h} h[l_2]\exp^{-\pi jl_2}=\sum_{l_2=0}^{L_h} (-1)^{l2}~h[l_2]\\
\end{equation}

And, we can deduce that $H[0]\neq H[N/2]$ for at least $l_1\neq l_2$

\end{enumerate}

\newpage


\section{Frequency selectivity of Wireless Channel}

\subsection{How is the impact of a frequency offset in OFDM systems different from that of single carrier systems? In particular, how is the impact on the signal constellation different?}

\subsection{What is the subcarrier spacing $\Delta_c$ of your system when $N = 1024$ and $64$ respectively?}

We use the following equation
\begin{equation}
\Delta_c=\frac{1}{NT}
\end{equation}
with T the sample period and N the FFT size. Finally, we obtain the following result

\begin{center}
	\begin{tabular}{c|c|c}
		  & $N=64$ & $N=1024$\\
		  \hline
	$\Delta_c$ & $\textbf{15625~kHz}$ & $\textbf{976.56 kHz}$ \\
	\end{tabular}
	\label{tab3}
\end{center}

\subsection{Which of the systems (i.e., $N = 1024$ or $64$) is more severely impacted by a $200 Hz$ frequency offset? Why?}






\subsection{Discuss this relationship between the effect of symbol timing error in single carrier systems and frequency offset in OFDM systems.}



\section{More question}

\subsection{What is the effective data rate of the OFDM system in lab?}

\subsection{Discuss why or why not you might use an OFDM system in a wideband and narrowband system respectively.}

\subsection{Name at least three parameters of OFDM systems which contribute to this flexibility and comment on how they do so.}


































\begin{comment}
\begin{figure}[!ht]
    \begin{minipage}[b]{0.48\linewidth}
        \centering \includegraphics[scale=0.45]{img/Sliding_correletaion_OFF_AWGN_5dB_shift_bit_4.PNG}
     \caption{Error rate for a delay+4 in AWGN}
     \label{fig5}
    \end{minipage}\hfill
    \begin{minipage}[b]{0.48\linewidth}
         \centering \includegraphics[scale=0.45]{img/Sliding_correletaion_OFF_AWGN_5dB_shift_bit_5.PNG}
 \caption{Error rate for a delay+5 in AWGN}\label{fig6}
    \end{minipage}
\end{figure}
\end{comment}


%\begin{figure}[!ht]
%\centering
%\includegraphics[scale=0.7]{img/test_offset_201hz_OFF.PNG}
%\caption{Constellation for a frequency offset of $201 \si{\hertz}$ without correction}
%\label{freq_correct_off}
%\end{figure}

%\begin{center}
%	\begin{tabular}{c|c}
%		Run & Frequency offset [\si{\hertz}]\\
%		\hline
%	\end{tabular}
%\end{center}


\end{document}
